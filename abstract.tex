\pagebreak
\chapter*{ABSTRACT}

An off-line Nepali handwriting recognition, based on the neural networks, is described in this research work. For the recognition of off-line handwritings with high classification rate a good set of features as a descriptor of image is required. Two important categories of the features are described, geometric and statistical features for extracting information from character images. Directional features are extracted from geometry of skeletonized character image and statistical features are extracted from the pixel distribution of skeletonized character image. The research primarily concerned with the problem of isolated handwritten character recognition for Nepali language. \ac{mlp} \& \ac{rbf} classifiers are used for classification. The principal contributions presented here are preprocessing, feature extraction and \ac{mlp} \& \ac{rbf} classifiers. The another important contribution is the creation of benchmark dataset for off-line Nepali handwritings. There are three datasets for Nepali handwritten numerals, Nepali handwritten vowels and Nepali handwritten consonants respectively. Nepali handwritten numeral dataset contains total $288$ samples for each $10$ classes of Nepali numerals, Nepali handwritten vowel dataset contains $221$ samples for each $12$ classes of Nepali vowels and Nepali handwritten consonant dataset contains $205$ samples for each $36$ classes of Nepali consonants. The strength of this research is efficient feature extraction and the comprehensive classification schemes due to which, the recognition accuracy of $\textbf{94.44\%}$ is obtained for Nepali handwritten numeral dataset, $\textbf{86.04\%}$ is obtained for Nepali handwritten vowel dataset and $\textbf{80.25\%}$ is obtained for Nepali handwritten consonant dataset.\par


{\textbf{Keywords:}\par\textit{Off-line handwriting recognition, Image processing, Neural networks, Multilayer perceptron, Radial basis function, Preprocessing, Feature extraction, Nepali handwritten datasets}
