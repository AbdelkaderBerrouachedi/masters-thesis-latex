\chapter{Sample Source Codes}
\section{Image Preprocessing Source Code}

\subsection*{Read Image}
\begin{lstlisting}
image=imread('image.jpg');
figure('units','normalized','outerposition',[0 0 1 1]);
imshow(image);
\end{lstlisting}

\subsection*{Colour Normalization}
\begin{lstlisting}
figure('units','normalized','outerposition',[0 0 0.5 0.5]);
subplot(1,2,1),imshow(image_rgb);
title('RGB Image','FontSize',14);
image_gray=rgb2gray(image_rgb);
subplot(1,2,2),imshow(image_gray);
title('Gray Image','FontSize',14);
\end{lstlisting}

\subsection*{Noise Removal}
\begin{lstlisting}
image_denoised=medfilt2(image_gray);
image_denoised=image_denoised(2:end-1,2:end-1); % remove corner pixels
subplot(1,2,1),imshow(image_gray);
title('Gray Image','FontSize',14);
subplot(1,2,2),imshow(image_denoised);
title('Denoised Image','FontSize',14);
\end{lstlisting}

\subsection*{Image Segmentation}
\begin{lstlisting}
threshold=graythresh(image); % Otsu's method for finding global threshold
image_binarized=im2bw(image_denoised,threshold);
subplot(1,2,1),imshow(image_denoised);
title('Denoised Image','FontSize',14);
subplot(1,2,2),imshow(image_binarized);
title('Binarized Image','FontSize',14);
\end{lstlisting}

\subsection*{Image Segmentation}
\begin{lstlisting}
image_inverted=~image_binarized;
subplot(1,2,1),imshow(image_binarized);
title('Binarized Image','FontSize',14);
subplot(1,2,2),imshow(image_inverted);
title('Inverted Image','FontSize',14);
\end{lstlisting}

\subsection*{Universe of Discourse Selection}
\begin{lstlisting}
image_discoursed=universe_of_discourse(image_inverted);
subplot(1,2,1),imshow(image_inverted);
title('Inverted Image','FontSize',14);
subplot(1,2,2),imshow(image_discoursed);
title('Discoursed Image','FontSize',14);
\end{lstlisting}

\subsection*{Image Size Normalization}
\begin{lstlisting}
image_normalized=imresize(image_discoursed,[36 36]);
subplot(1,2,1),imshow(image_discoursed);
title('Discoursed Image','FontSize',14);
subplot(1,2,2),imshow(image_normalized);
title('Size Normalized Image (36x36)','FontSize',14);
\end{lstlisting}

\subsection*{Image Skeletonization}
\begin{lstlisting}
image_skeletonized=bwmorph(image_normalized,'skel','inf');
subplot(1,2,1),imshow(image_normalized);
title('Size Normalized Image (36x36)','FontSize',14);
subplot(1,2,2),imshow(image_skeletonized);
title('Skeletonized Image','FontSize',14);
\end{lstlisting}

\section{Feature Extraction Source Code}
\subsection*{Directional Features}
\begin{lstlisting}
features=[]; % initializing local feature vector
% dividing given image into 3x3 zones
[row,col]=size(image);
zone_height=row/3; %12 pixels (because image size is normalized to 36x36)
zone_width=col/3; %12 pixels (because image size is normalized to 36x36)

zone11=image(1:zone_height,1:zone_width);
zone12=image(1:zone_height,(zone_width+1):2*zone_width);
zone13=image(1:zone_height,(2*zone_width+1):end);

zone21=image((zone_height+1):2*zone_height,1:zone_width);
zone22=image((zone_height+1):2*zone_height,(zone_width+1):2*zone_width);
zone23=image((zone_height+1):2*zone_height,(2*zone_width+1):end);

zone31=image((2*zone_height+1):end,1:zone_width);
zone32=image((2*zone_height+1):end,(zone_width+1):2*zone_width);
zone33=image((2*zone_height+1):end,(2*zone_width+1):end);

% directional features
zone11_features=lineclassifier(zone11);
zone12_features=lineclassifier(zone12);
zone13_features=lineclassifier(zone13);

zone21_features=lineclassifier(zone21);
zone22_features=lineclassifier(zone22);
zone23_features=lineclassifier(zone23);

zone31_features=lineclassifier(zone31);
zone32_features=lineclassifier(zone32);
zone33_features=lineclassifier(zone33);

% Combining features from all nine zones of given image.
%Each row of feature vector contains nine geometric feature elements.
features=[zone11_features;zone12_features;zone13_features;
zone21_features;zone22_features;zone23_features;zone31_features;
zone32_features;zone33_features];
% 'feature' vector is 81 dimensional row vector. 
% As there are 9 zones with 9 feature elements each.

% Display zonned images with corresponding feature vectors figure;
subplot(3,3,1),imshow(zone11);title('zone11','FontSize',12);
subplot(3,3,2),imshow(zone12);title('zone12','FontSize',12);
subplot(3,3,3),imshow(zone13);title('zone13','FontSize',12);
subplot(3,3,4),imshow(zone21);title('zone21','FontSize',12);
subplot(3,3,5),imshow(zone22);title('zone22','FontSize',12);
subplot(3,3,6),imshow(zone23);title('zone23','FontSize',12);
subplot(3,3,7),imshow(zone31);title('zone31','FontSize',12);
subplot(3,3,8),imshow(zone32);title('zone32','FontSize',12);
subplot(3,3,9),imshow(zone33);title('zone33','FontSize',12);
\end{lstlisting}

\subsection*{Moment Invariant Features}
\begin{lstlisting}
% Normalized central moments are calculated from each image. These regional
% descriptors form seven elememt feature vector as there are seven moments
% invarients. Normalized Central Moments are independ of transalation,
% rotation, scalling and can be used in pattern identification.
image_size=numel(image); % Number of elemets in given image (36x36=1296)
moments=invmoments(image); % 7 dimensional column vector
\end{lstlisting}

\subsection*{Euler Number Feature}
\begin{lstlisting}
euler_number_feature=bweuler(image);
\end{lstlisting}

\subsection*{Area of Character Skeleton Feature}
\begin{lstlisting}
area=sum(sum(image));
\end{lstlisting}

\subsection*{Centroid Features}
\begin{lstlisting}
[rows,cols] = size(image);
x = ones(rows,1)*[1:cols]; %Matrix with each pixel set to its x coordinate
y = [1:rows]'*ones(1,cols); % "     "     "    "    "  "   "  y    "

area = sum(sum(image));
meanx = sum(sum(double(image).*x))/area;
meany = sum(sum(double(image).*y))/area;

centroid.x=meanx;
centroid.y=meany;

centroid_features=[centroid.x centroid.y]';
\end{lstlisting}

\subsection*{Eccentricity Features}
\begin{lstlisting}
pixellist=regionprops(image,'pixellist') ;

total_list=[];
for i=1:numel(pixellist)
total_list=[total_list;pixellist(i).PixelList];
end
list=total_list;

if (isempty(list))
MajorAxisLength = 0;
MinorAxisLength = 0;
Eccentricity = 0;
else
  % Assign X and Y variables so that we're measuring orientation
  % counterclockwise from the horizontal axis.

  xbar = centroid.x;
  ybar = centroid.y;

  x = list(:,1) - xbar;
  y = -(list(:,2) - ybar); % This is negative for the
  % orientation calculation (measured in the
  % counter-clockwise direction).

  N = length(x);

  % Calculate normalized second central moments for the region. 1/12 is
  % the normalized second central moment of a pixel with unit length.
  uxx = sum(x.^2)/N + 1/12;
  uyy = sum(y.^2)/N + 1/12;
  uxy = sum(x.*y)/N;

  % Calculate major axis length, minor axis length, and eccentricity.
  common = sqrt((uxx - uyy)^2 + 4*uxy^2);
  MajorAxisLength = 2*sqrt(2)*sqrt(uxx + uyy + common);
  MinorAxisLength = 2*sqrt(2)*sqrt(uxx + uyy - common);
  Eccentricity = 2*sqrt((MajorAxisLength/2)^2 -
      (MinorAxisLength/2)^2) / MajorAxisLength;

e=Eccentricity;
end
\end{lstlisting}
\section{Training and Testing Source Code}
\subsection*{MLP Training and Testing}
\begin{lstlisting}
% Initialize network parameters and proceeds training and testing
inputs = input_feature_vector;
targets = target_feature_vector;

% Mapping all the input vectors to the range [-1 1]
% [inputs,input_maping_setting]=mapminmax(inputs);
% [targets,target_maping_setting]=mapminmax(targets);


% Create a Feedforward Pattern Recognition Network
% Feedforward network with one input layer, one hidden layer and
% one output layer.

hiddenSizes=25; % Number of neurons in hidden layer

net = feedforwardnet(hiddenSizes);

% Setup Division of Data for Training, Validation, Testing
Q=size(input_feature_vector,2); % Number of image samples
trainRatio = 70/100;    % training samples
valRatio =   15/100;    % validation samples
testRatio = 15/100;     % testing samples

[trainInd,valInd,testInd] = dividerand(Q,trainRatio,valRatio,testRatio);

% Select training and testing samples
% training samples
train_feature_vector=[];
train_target_vector=[];
idx=1;
for i=1:size(trainInd,2)
	train_feature_vector(:,idx)=input_feature_vector(:,trainInd(i));
	train_target_vector(:,idx)=target_feature_vector(:,trainInd(i));
	idx=idx+1;
end

% testing samples
test_feature_vector=[];
test_target_vector=[];
idx=1;
for i=1:size(testInd,2)
	test_feature_vector(:,idx)=input_feature_vector(:,testInd(i));
	test_target_vector(:,idx)=target_feature_vector(:,testInd(i));
	idx=idx+1;
end


% Choose a Network Training Function
   net.trainFcn = 'trainlm';  % Levenberg-Marquardt Learning Algorithm
%  Function Parameters for 'trainlm'
%
%     Show Training Window Feedback   showWindow: true
%     Show Command Line Feedback showCommandLine: false
%     Command Line Frequency                show: 20
%     Maximum Epochs                      epochs: 5000
%     Maximum Training Time                 time: Inf
%     Performance Goal                      goal: 0.1
%     Minimum Gradient                  min_grad: 1e-005
%     Maximum Validation Checks         max_fail: 6
%     Mu                                      mu: 0.001
%     Mu Decrease Ratio                   mu_dec: 0.1
%     Mu Increase Ratio                   mu_inc: 10
%     Maximum mu                          mu_max: 10000000000

% Choose Network Weight Initialization  Function
net.layers{1}.initFcn= 'initnw'; % Nguyen Widrow weight initialization
net.layers{2}.initFcn= 'initnw';

% Choose Network Transfer Function
net.layers{1}.transferFcn='tansig'; % a = tansig(n) = 2/(1+exp(-2*n))-1
net.layers{2}.transferFcn='tansig';

% Choose a Performance Function
net.performFcn = 'mse';  % Mean squared error

% Choose a network goal (minimum mean square error)
net.trainParam.goal = 1e-003;
% Choose  time imterval to show the network status
net.trainParam.show = 20;
% Choose number of Epoches (iterations)
net.trainParam.epochs = 1000;
% Choose value of mu
net.trainParam.mu=0.001;
% memory reduction parameter
net.efficiency.memoryReduction=1;
% Choose Plot Functions
% For a list of all plot functions type: help nnplot
net.plotFcns = {'plotperform','plottrainstate','ploterrhist', ...
  'plotregression', 'plotfit','plotconfusion'};

% Train the Network
[net,tr] = train(net,train_feature_vector,train_target_vector);

% Test the Network
outputs = sim(net,test_feature_vector);
errors = gsubtract(test_target_vector,outputs);
performance = perform(net,test_target_vector,outputs);

% Overall Percentage of correct and incorrect classification
[c,cm,ind,per] = confusion(test_target_vector,outputs);
% c     Confusion value = fraction of samples misclassified
% cm	S-by-S confusion matrix, where cm(i,j) is the number 
% of samples whose target is the ith class that was classified as j

fprintf('Percentage Correct Classification   : %f%%\n', 100*(1-c))
fprintf('Percentage Incorrect Classification : %f%%\n', 100*c)
fprintf('Time to Simulate the Network : %0.2f sec \n', tr.time(end))

save mlp_trained_network;
\end{lstlisting}

\subsection*{RBF Training and Testing}
\begin{lstlisting}
% Initialize network parameters and proceeds training and testing
inputs = input_feature_vector;
targets = target_feature_vector;


% Mapping all the input vectors to the range [-1 1]
% [inputs,input_maping_setting]=mapminmax(inputs);
% [targets,target_maping_setting]=mapminmax(targets);

% Setup Division of Data for Training, Validation, Testing
Q=size(input_feature_vector,2); % Number of image samples
trainRatio = 70/100;    % training samples
valRatio =   15/100;    % validation samples
testRatio = 15/100;     % testing samples

[trainInd,valInd,testInd] = dividerand(Q,trainRatio,valRatio,testRatio);

% Select training and testing samples 
% training samples
train_feature_vector=[];
train_target_vector=[];
idx=1;
for i=1:size(trainInd,2)
	train_feature_vector(:,idx)=input_feature_vector(:,trainInd(i));
	train_target_vector(:,idx)=target_feature_vector(:,trainInd(i));
	idx=idx+1;
end

% testing samples
test_feature_vector=[];
test_target_vector=[];
idx=1;
for i=1:size(testInd,2)
	test_feature_vector(:,idx)=input_feature_vector(:,testInd(i));
	test_target_vector(:,idx)=target_feature_vector(:,testInd(i));
	idx=idx+1;
end

% Initialize the Radial Basis Function Parameters

goal=0.01;  % default=0.0
spread=2.0; % default =1.0
max_neurons=size(train_target_vector,2);
neuron_increment=25;

% Creating Radial Basis Network
tic;
net=newrb(train_feature_vector,train_target_vector,
		  goal,spread,max_neurons,neuron_increment);
time=toc;

% -----------Testing the network---------------

outputs=sim(net,test_feature_vector);
errors = gsubtract(test_target_vector,outputs);
% Overall Percentage of correct and incorrect classification
[c,cm,ind,per] = confusion(test_target_vector,outputs);
% c     Confusion value = fraction of samples misclassified
% cm	S-by-S confusion matrix, where cm(i,j) is the number
% of samples whose target is the ith class that was classified as j

fprintf('Percentage Correct Classification   : %f%%\n', 100*(1-c));
fprintf('Percentage Incorrect Classification : %f%%\n', 100*c);
fprintf('Time to Simulate the Network : %0.2f sec \n', time);
% --------------------------------------------------------

save rbf_trained_network;
\end{lstlisting}